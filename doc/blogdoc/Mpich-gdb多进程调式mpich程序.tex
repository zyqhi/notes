%%%%%%%%%%%%%%%%%%%%%%%%%%%%%%%%%%%%%%%%%
%
%    作者:周永强
%
%%%%%%%%%%%%%%%%%%%%%%%%%%%%%%%%%%%%%%%%%

\documentclass[a4paper]{article}

\usepackage{zhfontcfg} % 中文支持包,字体定义等
\usepackage{geometry} % 使得用这个宏包,减少两边的白边,使得看上去更紧凑
\usepackage{indentfirst} % 中文首行缩进
\usepackage{xcolor} % 使用颜色
\usepackage{verbments} % 用于插入代码

\linespread{1} % 设置行距,这里为单倍行距,也可以设置其他值改变行距
\setlength{\parskip}{0.5\baselineskip} % 设置段落间距
%\setmonofont{AR PL UKai CN} % 中文等宽,不然代码里面中文乱码

%----------------------------------------------------------------------------------------
%    文档信息
%----------------------------------------------------------------------------------------

\title{使用gdb调试mpich多进程程序} % 文档名称
\author{周永强} % 作者姓名
\date{\today} % 文档日期

\begin{document}

\maketitle % 插入文档标题

\section{说明} % 可以为摘要

\subsection{目的} % 是什么
本文档讲述如何在linux下通过gdb对mpich程序进行调试,mpich运行通常是多个进程同时运行,因此调试起来比较麻烦,而且具有一定的技巧。文中通过一个调试例子展示如何进行调试。

\subsection{原理} % 为什么
调试的方式有两个要点:一是通过进程号来进入要调试的程序,二是通过在源代码中设置循环的技巧使程序阻塞。

\section{调试过程} % 怎么做
要调试的程序为hellow,其源代码为hello.c。
\begin{enumerate} % 序列
  \item 编辑hello.c的源文件如下
    \begin{pyglist}[language=c]
#include <stdio.h>
#include "mpi.h"

#define MAX_STRING 100
 
int main( int argc, char *argv[] )
{
    char greeting[MAX_STRING];
    int my_rank;
    int comm_sz;
    int a = 1;
    
    MPI_Init(NULL, NULL);
    MPI_Comm_rank(MPI_COMM_WORLD, &my_rank);
    MPI_Comm_size(MPI_COMM_WORLD, &comm_sz);
    
    if (my_rank == 0)
    {
        printf("I am root.\n");
        sprintf(greeting, "Hello, my child", my_rank, comm_sz);
        
        while (a == 1);
        MPI_Send(greeting, strlen(greeting)+1,
          MPI_CHAR, 1, 0, MPI_COMM_WORLD);
    }
    else
    {
        while (a == 1);
        MPI_Recv(greeting, MAX_STRING, MPI_CHAR,
          0, 0, MPI_COMM_WORLD, MPI_STATUS_IGNORE);
        printf("I am child: %d, greeting from root: %s\n",
          my_rank, greeting);
    }
    
    MPI_Finalize();
    return 0;
}
    \end{pyglist}

  \item 编译程序,注意加上参数-g
    \begin{pyglist}[language=bash]
      mpicc -g hellow.c -o hellow
    \end{pyglist}

  \item 运行hellow,这里以两个进程为例
    \begin{pyglist}[language=bash]
      mpiexec -n 2 ./hellow
    \end{pyglist}
  
 \item 由于设置了死循环,所以上述程序会阻塞执行,另外打开两个终端,分别对两个进程进行调试,调试前需要先获取进程的pid号,获取命令为
   \begin{pyglist}[language=bash]
     ps aux | grep hellow
   \end{pyglist}
 
 \item 找到对应的进程号,我机器上hellow对应的两个进程分别为9135和9136,可以对二者分别进行调试,调试9135号进程的方式为,首先运行gdb,命令为
   \begin{pyglist}[language=bash]
     gdb
   \end{pyglist}
   
 \item gdb启动以后,对9135进程调试,9136调试方式一样,在gdb中运行命令
   \begin{pyglist}[language=bash]
     attach 9135
   \end{pyglist}
   
 \item 此时可以看到程序阻塞在循环处,通过设置循环的变量,使循环的条件变为假,跳出循环,进入下一步的执行,便可接着进行调试了
   \begin{pyglist}[language=bash]
     set var a = 0
   \end{pyglist}
   
\end{enumerate}

\end{document}








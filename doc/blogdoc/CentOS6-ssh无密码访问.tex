%%%%%%%%%%%%%%%%%%%%%%%%%%%%%%%%%%%%%%%%%
% 通用blog指南文档格式
% LaTeX Template
% Version 1.0 (6/20/2014)
%
% 作者:周永强
%
%
%%%%%%%%%%%%%%%%%%%%%%%%%%%%%%%%%%%%%%%%%

\documentclass[a4paper]{article}

\usepackage{zhfontcfg} % 中文支持包,字体定义等
\usepackage{geometry} % 使得用这个宏包,减少两边的白边,使得看上去更紧凑
\usepackage{indentfirst} % 中文首行缩进
\usepackage{xcolor} % 使用颜色
\usepackage{verbments} % 用于插入代码

\linespread{1} % 设置行距,这里为单倍行距,也可以设置其他值改变行距
\setlength{\parskip}{0.5\baselineskip} % 设置段落间距
%\setmonofont{AR PL UKai CN} % 中文等宽,不然代码里面中文乱码

%----------------------------------------------------------------------------------------
%    文档信息
%----------------------------------------------------------------------------------------

\title{CentOS6 ssh配置无密码访问} % 文档名称
\author{周永强} % 作者姓名
\date{\today} % 文档日期

\begin{document}

\maketitle % 插入文档标题

\section{说明} % 可以为摘要

\subsection{目的} % 是什么
本文档讲述如何设置ssh无密码访问,环境为两台装有CentOS 6.4的计算机,每台计算机至少有一块以太网卡,二者可以通过网络相互访问,即通过主机名可以相互ping通,二者的主机名分别为node01和node02。
\begin{pyglist}[language=bash]
   root@node01表示: 位于主机node01上的root用户。
   root@node01表示: 位于主机node02上的root用户。
\end{pyglist}
更准备的表述为:通过设置ssh使得root@node01可以无密码访问root@node02。

\subsection{原理} % 为什么
root@node01产生自己的公钥和私钥,并将自己的公钥上传到root@node02的信任文件里面,这样root@node01用户每次通过ssh登录到root@node02时便可通过rsa进行验证,免去输入密码的麻烦。

\section{配置过程} % 怎么做

\begin{enumerate} % 序列
  \item root@node02 : 编辑/etc/ssh/sshd\_config文件,去掉如下条目的内容前的注释符号\#
    \begin{pyglist}[language=bash]
      RSAAuthentication yes
      PubkeyAuthentication yes
      AuthorizedKeysFile     .ssh/authorized_keys
    \end{pyglist}

  \item root@node02 : 重启sshd服务
    \begin{pyglist}[language=bash]
      service sshd restart
    \end{pyglist}

  \item root@node01 : 生成自己的密钥和私钥,直接enter按照默认设置即可,不需要输入任何内容,正常情况下这一步执行完成后,应该在~/.ssh目录下生成两个文件id\_rsa(私钥)和id\_rsa.pub(公钥)
    \begin{pyglist}[language=bash]
      ssh-keygen -t rsa
    \end{pyglist}
  
 \item root@node01 : 将自己的公钥加入root@node02的信任区,并设置权限
   \begin{pyglist}[language=bash]
     ssh root@node02 'mkdir -p /root/.ssh'
     scp /root/.ssh/id_rsa.pub root@node02:/root/.ssh/authorized_keys
     ssh root@node02 'chmod  700 /root/.ssh'
     ssh root@node02 'chmod  600 /root/.ssh/*'
   \end{pyglist}
 
 \item root@node01 : 完成上述步骤已经可以无密码访问了,但是CentOS 6.4比较乖张,在其Bug Tracker上看到导致原因是当设置selinux为enforcing时,所有客户端的认证都被忽略。解决方法是,运行下列命令。
   \begin{pyglist}[language=bash]
     ssh root@node02 'restorecon -R -v /root/.ssh'
   \end{pyglist}
\end{enumerate}

\end{document}

